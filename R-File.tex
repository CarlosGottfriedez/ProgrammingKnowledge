\documentclass[]{article}
\usepackage[T1]{fontenc}
\usepackage{lmodern}
\usepackage{amssymb,amsmath}
\usepackage{ifxetex,ifluatex}
\usepackage{fixltx2e} % provides \textsubscript
% use upquote if available, for straight quotes in verbatim environments
\IfFileExists{upquote.sty}{\usepackage{upquote}}{}
\ifnum 0\ifxetex 1\fi\ifluatex 1\fi=0 % if pdftex
  \usepackage[utf8]{inputenc}
\else % if luatex or xelatex
  \ifxetex
    \usepackage{mathspec}
    \usepackage{xltxtra,xunicode}
  \else
    \usepackage{fontspec}
  \fi
  \defaultfontfeatures{Mapping=tex-text,Scale=MatchLowercase}
  \newcommand{\euro}{€}
\fi
% use microtype if available
\IfFileExists{microtype.sty}{\usepackage{microtype}}{}
\usepackage[margin=1in]{geometry}
\usepackage{color}
\usepackage{fancyvrb}
\newcommand{\VerbBar}{|}
\newcommand{\VERB}{\Verb[commandchars=\\\{\}]}
\DefineVerbatimEnvironment{Highlighting}{Verbatim}{commandchars=\\\{\}}
% Add ',fontsize=\small' for more characters per line
\usepackage{framed}
\definecolor{shadecolor}{RGB}{48,48,48}
\newenvironment{Shaded}{\begin{snugshade}}{\end{snugshade}}
\newcommand{\KeywordTok}[1]{\textcolor[rgb]{0.94,0.87,0.69}{{#1}}}
\newcommand{\DataTypeTok}[1]{\textcolor[rgb]{0.87,0.87,0.75}{{#1}}}
\newcommand{\DecValTok}[1]{\textcolor[rgb]{0.86,0.86,0.80}{{#1}}}
\newcommand{\BaseNTok}[1]{\textcolor[rgb]{0.86,0.64,0.64}{{#1}}}
\newcommand{\FloatTok}[1]{\textcolor[rgb]{0.75,0.75,0.82}{{#1}}}
\newcommand{\CharTok}[1]{\textcolor[rgb]{0.86,0.64,0.64}{{#1}}}
\newcommand{\StringTok}[1]{\textcolor[rgb]{0.80,0.58,0.58}{{#1}}}
\newcommand{\CommentTok}[1]{\textcolor[rgb]{0.50,0.62,0.50}{{#1}}}
\newcommand{\OtherTok}[1]{\textcolor[rgb]{0.94,0.94,0.56}{{#1}}}
\newcommand{\AlertTok}[1]{\textcolor[rgb]{1.00,0.81,0.69}{{#1}}}
\newcommand{\FunctionTok}[1]{\textcolor[rgb]{0.94,0.94,0.56}{{#1}}}
\newcommand{\RegionMarkerTok}[1]{\textcolor[rgb]{0.80,0.80,0.80}{{#1}}}
\newcommand{\ErrorTok}[1]{\textcolor[rgb]{0.76,0.75,0.62}{{#1}}}
\newcommand{\NormalTok}[1]{\textcolor[rgb]{0.80,0.80,0.80}{{#1}}}
\usepackage{graphicx}
% Redefine \includegraphics so that, unless explicit options are
% given, the image width will not exceed the width of the page.
% Images get their normal width if they fit onto the page, but
% are scaled down if they would overflow the margins.
\makeatletter
\def\ScaleIfNeeded{%
  \ifdim\Gin@nat@width>\linewidth
    \linewidth
  \else
    \Gin@nat@width
  \fi
}
\makeatother
\let\Oldincludegraphics\includegraphics
{%
 \catcode`\@=11\relax%
 \gdef\includegraphics{\@ifnextchar[{\Oldincludegraphics}{\Oldincludegraphics[width=\ScaleIfNeeded]}}%
}%
\ifxetex
  \usepackage[setpagesize=false, % page size defined by xetex
              unicode=false, % unicode breaks when used with xetex
              xetex]{hyperref}
\else
  \usepackage[unicode=true]{hyperref}
\fi
\hypersetup{breaklinks=true,
            bookmarks=true,
            pdfauthor={},
            pdftitle={R-Funktionen},
            colorlinks=true,
            citecolor=blue,
            urlcolor=blue,
            linkcolor=magenta,
            pdfborder={0 0 0}}
\urlstyle{same}  % don't use monospace font for urls
\setlength{\parindent}{0pt}
\setlength{\parskip}{6pt plus 2pt minus 1pt}
\setlength{\emergencystretch}{3em}  % prevent overfull lines
\setcounter{secnumdepth}{5}

%%% Change title format to be more compact
\usepackage{titling}
\setlength{\droptitle}{-2em}
  \title{R-Funktionen}
  \pretitle{\vspace{\droptitle}\centering\huge}
  \posttitle{\par}
  \author{}
  \preauthor{}\postauthor{}
  \date{}
  \predate{}\postdate{}




\begin{document}

\maketitle


\section{R-Funktionen}\label{r-funktionen}

Diese Seite listet all die R-Funktionen auf die ich schonmal benutzt
habe. Ich versuche so weit es geht die ausgeführten Beispiele mit
anzuzeigen. ***

Mit ? können wird die Dokumentation von R aufgerufen und weitere
zusätzliche Informationen werden angezeigt.

\begin{Shaded}
\begin{Highlighting}[]
\CommentTok{# zum Beispiel}
\NormalTok{?row.names}
\end{Highlighting}
\end{Shaded}

\begin{verbatim}
c, nchar, data, str, dim, names, head, and tail.
\end{verbatim}

\begin{Shaded}
\begin{Highlighting}[]
\CommentTok{# zeigt alle Observations an}
\KeywordTok{row.names}\NormalTok{(mtcars)}
\end{Highlighting}
\end{Shaded}

\begin{verbatim}
##  [1] "Mazda RX4"           "Mazda RX4 Wag"       "Datsun 710"         
##  [4] "Hornet 4 Drive"      "Hornet Sportabout"   "Valiant"            
##  [7] "Duster 360"          "Merc 240D"           "Merc 230"           
## [10] "Merc 280"            "Merc 280C"           "Merc 450SE"         
## [13] "Merc 450SL"          "Merc 450SLC"         "Cadillac Fleetwood" 
## [16] "Lincoln Continental" "Chrysler Imperial"   "Fiat 128"           
## [19] "Honda Civic"         "Toyota Corolla"      "Toyota Corona"      
## [22] "Dodge Challenger"    "AMC Javelin"         "Camaro Z28"         
## [25] "Pontiac Firebird"    "Fiat X1-9"           "Porsche 914-2"      
## [28] "Lotus Europa"        "Ford Pantera L"      "Ferrari Dino"       
## [31] "Maserati Bora"       "Volvo 142E"
\end{verbatim}

\begin{center}\rule{0.5\linewidth}{\linethickness}\end{center}

\textbf{Es folgen mehrere Beispiel mit R-Code}

\begin{Shaded}
\begin{Highlighting}[]
\CommentTok{# Gibt die Werte der Variable mpg des Datenframes mtcars aus. (Also $ als Symbol).}
\NormalTok{mtcars$mpg}
\end{Highlighting}
\end{Shaded}

\begin{verbatim}
##  [1] 21.0 21.0 22.8 21.4 18.7 18.1 14.3 24.4 22.8 19.2 17.8 16.4 17.3 15.2
## [15] 10.4 10.4 14.7 32.4 30.4 33.9 21.5 15.5 15.2 13.3 19.2 27.3 26.0 30.4
## [29] 15.8 19.7 15.0 21.4
\end{verbatim}

\subsection{Funktionen die das Zentrum
beschreiben}\label{funktionen-die-das-zentrum-beschreiben}

\begin{Shaded}
\begin{Highlighting}[]
\CommentTok{# Gibt den Mittelwert der Daten an.}
\KeywordTok{mean}\NormalTok{(mtcars$mpg)  }
\end{Highlighting}
\end{Shaded}

\begin{verbatim}
## [1] 20.09
\end{verbatim}

\begin{Shaded}
\begin{Highlighting}[]
\CommentTok{# Gibt den Median der Daten an.}
\KeywordTok{median}\NormalTok{(mtcars$mpg)}
\end{Highlighting}
\end{Shaded}

\begin{verbatim}
## [1] 19.2
\end{verbatim}

\begin{Shaded}
\begin{Highlighting}[]
\CommentTok{# Überblick über alle Daten}
\KeywordTok{summary}\NormalTok{(cars)}
\end{Highlighting}
\end{Shaded}

\begin{verbatim}
##      speed           dist    
##  Min.   : 4.0   Min.   :  2  
##  1st Qu.:12.0   1st Qu.: 26  
##  Median :15.0   Median : 36  
##  Mean   :15.4   Mean   : 43  
##  3rd Qu.:19.0   3rd Qu.: 56  
##  Max.   :25.0   Max.   :120
\end{verbatim}

\subsection{Funktionen um Datenframes zu
laden}\label{funktionen-um-datenframes-zu-laden}

\begin{Shaded}
\begin{Highlighting}[]
\CommentTok{#Zeigt das Verzeichnis an in welchen wir uns befinden.}
\KeywordTok{getwd}\NormalTok{() }
\end{Highlighting}
\end{Shaded}

\begin{verbatim}
## [1] "/Users/user/AllGitHub/ProgrammingKnowledge"
\end{verbatim}

\begin{verbatim}
#Wechsel des Verzeichnises setwd('~/Downloads') immer in **'' - Zeichen**.
setwd('~/Downloads') 
#cvs-Datei einlesen.
read.csv('reddit.csv')
\end{verbatim}

\subsection{Andere Funktionen}\label{andere-funktionen}

\begin{Shaded}
\begin{Highlighting}[]
\CommentTok{# zeigt ein Subset}
\KeywordTok{subset}\NormalTok{(mtcars, mtcars$mpg<=}\DecValTok{25} \NormalTok{&}\StringTok{ }\NormalTok{mtcars$wt<=}\FloatTok{2.581} \NormalTok{)   }
\end{Highlighting}
\end{Shaded}

\begin{verbatim}
##                mpg cyl  disp hp drat    wt  qsec vs am gear carb
## Datsun 710    22.8   4 108.0 93 3.85 2.320 18.61  1  1    4    1
## Toyota Corona 21.5   4 120.1 97 3.70 2.465 20.01  1  0    3    1
\end{verbatim}

\begin{Shaded}
\begin{Highlighting}[]
\CommentTok{# Zeigt einen Überblick an}
\KeywordTok{summary}\NormalTok{(mtcars)}
\end{Highlighting}
\end{Shaded}

\begin{verbatim}
##       mpg            cyl            disp             hp       
##  Min.   :10.4   Min.   :4.00   Min.   : 71.1   Min.   : 52.0  
##  1st Qu.:15.4   1st Qu.:4.00   1st Qu.:120.8   1st Qu.: 96.5  
##  Median :19.2   Median :6.00   Median :196.3   Median :123.0  
##  Mean   :20.1   Mean   :6.19   Mean   :230.7   Mean   :146.7  
##  3rd Qu.:22.8   3rd Qu.:8.00   3rd Qu.:326.0   3rd Qu.:180.0  
##  Max.   :33.9   Max.   :8.00   Max.   :472.0   Max.   :335.0  
##       drat            wt            qsec            vs       
##  Min.   :2.76   Min.   :1.51   Min.   :14.5   Min.   :0.000  
##  1st Qu.:3.08   1st Qu.:2.58   1st Qu.:16.9   1st Qu.:0.000  
##  Median :3.69   Median :3.33   Median :17.7   Median :0.000  
##  Mean   :3.60   Mean   :3.22   Mean   :17.8   Mean   :0.438  
##  3rd Qu.:3.92   3rd Qu.:3.61   3rd Qu.:18.9   3rd Qu.:1.000  
##  Max.   :4.93   Max.   :5.42   Max.   :22.9   Max.   :1.000  
##        am             gear           carb     
##  Min.   :0.000   Min.   :3.00   Min.   :1.00  
##  1st Qu.:0.000   1st Qu.:3.00   1st Qu.:2.00  
##  Median :0.000   Median :4.00   Median :2.00  
##  Mean   :0.406   Mean   :3.69   Mean   :2.81  
##  3rd Qu.:1.000   3rd Qu.:4.00   3rd Qu.:4.00  
##  Max.   :1.000   Max.   :5.00   Max.   :8.00
\end{verbatim}

\begin{Shaded}
\begin{Highlighting}[]
\CommentTok{# neue Spalte mit year als überschrifft und 1974 für alle}
\NormalTok{mtcars$year <-}\StringTok{ }\DecValTok{1974}    
\end{Highlighting}
\end{Shaded}

\begin{Shaded}
\begin{Highlighting}[]
\CommentTok{# Spalte löschen}
\NormalTok{mtcars <-}\StringTok{ }\KeywordTok{subset}\NormalTok{(mtcars, }\DataTypeTok{select =} \NormalTok{-year)}
\end{Highlighting}
\end{Shaded}

\begin{Shaded}
\begin{Highlighting}[]
\CommentTok{# Ein Beispiel für Konditionen}
\NormalTok{mtcars$wt}
\end{Highlighting}
\end{Shaded}

\begin{verbatim}
##  [1] 2.620 2.875 2.320 3.215 3.440 3.460 3.570 3.190 3.150 3.440 3.440
## [12] 4.070 3.730 3.780 5.250 5.424 5.345 2.200 1.615 1.835 2.465 3.520
## [23] 3.435 3.840 3.845 1.935 2.140 1.513 3.170 2.770 3.570 2.780
\end{verbatim}

\begin{Shaded}
\begin{Highlighting}[]
\NormalTok{cond <-}\StringTok{ }\NormalTok{mtcars$wt <}\StringTok{ }\DecValTok{3}
\NormalTok{cond}
\end{Highlighting}
\end{Shaded}

\begin{verbatim}
##  [1]  TRUE  TRUE  TRUE FALSE FALSE FALSE FALSE FALSE FALSE FALSE FALSE
## [12] FALSE FALSE FALSE FALSE FALSE FALSE  TRUE  TRUE  TRUE  TRUE FALSE
## [23] FALSE FALSE FALSE  TRUE  TRUE  TRUE FALSE  TRUE FALSE  TRUE
\end{verbatim}

\begin{Shaded}
\begin{Highlighting}[]
\NormalTok{mtcars$weight_class <-}\StringTok{ }\KeywordTok{ifelse}\NormalTok{(cond, }\StringTok{'light'}\NormalTok{, }\StringTok{'average'}\NormalTok{)}
\NormalTok{mtcars$weight_class}
\end{Highlighting}
\end{Shaded}

\begin{verbatim}
##  [1] "light"   "light"   "light"   "average" "average" "average" "average"
##  [8] "average" "average" "average" "average" "average" "average" "average"
## [15] "average" "average" "average" "light"   "light"   "light"   "light"  
## [22] "average" "average" "average" "average" "light"   "light"   "light"  
## [29] "average" "light"   "average" "light"
\end{verbatim}

\begin{Shaded}
\begin{Highlighting}[]
\NormalTok{cond <-}\StringTok{ }\NormalTok{mtcars$wt >}\StringTok{ }\FloatTok{3.5}
\NormalTok{mtcars$weight_class <-}\StringTok{ }\KeywordTok{ifelse}\NormalTok{(cond, }\StringTok{'heavy'}\NormalTok{, mtcars$weight_class)}
\NormalTok{mtcars$weight_class}
\end{Highlighting}
\end{Shaded}

\begin{verbatim}
##  [1] "light"   "light"   "light"   "average" "average" "average" "heavy"  
##  [8] "average" "average" "average" "average" "heavy"   "heavy"   "heavy"  
## [15] "heavy"   "heavy"   "heavy"   "light"   "light"   "light"   "light"  
## [22] "heavy"   "average" "heavy"   "heavy"   "light"   "light"   "light"  
## [29] "average" "light"   "heavy"   "light"
\end{verbatim}

\begin{Shaded}
\begin{Highlighting}[]
\CommentTok{# entfernt code aus dem arbeitsbereich}
\KeywordTok{rm}\NormalTok{(cond)}
\KeywordTok{rm}\NormalTok{(efficient)}
\end{Highlighting}
\end{Shaded}

\begin{verbatim}
## Warning: Objekt 'efficient' nicht gefunden
\end{verbatim}

\begin{Shaded}
\begin{Highlighting}[]
\CommentTok{# zeigt die Anzahl der Fahrzeuge mit bestimmten Werten an}
\KeywordTok{table}\NormalTok{(mtcars$mpg)}
\end{Highlighting}
\end{Shaded}

\begin{verbatim}
## 
## 10.4 13.3 14.3 14.7   15 15.2 15.5 15.8 16.4 17.3 17.8 18.1 18.7 19.2 19.7 
##    2    1    1    1    1    2    1    1    1    1    1    1    1    2    1 
##   21 21.4 21.5 22.8 24.4   26 27.3 30.4 32.4 33.9 
##    2    2    1    2    1    1    1    2    1    1
\end{verbatim}

\begin{verbatim}
# Für Faktrone als Datentypen
levels(reddit$age.range)
\end{verbatim}

\subsection{qplot}\label{qplot}

qplot is the basic plotting function in the ggplot2 package, designed to
be familiar if you're used to plot from the base package. Parameter für
qplot:

\texttt{x = Variabelname} \texttt{data = Datenname}
\texttt{xlim = Vektor(Von, Bis)} \texttt{binwidth = Balkendicke}
\texttt{facet\_wrap(\textasciitilde{}gender, ncol = 2) -\/- Aufteilen in Einzelne kleine Fenster}

\subsubsection{Um eine Plot zu zeichnen}\label{um-eine-plot-zu-zeichnen}

\begin{Shaded}
\begin{Highlighting}[]
\CommentTok{#install.packages('ggplot2', dependencies = T)}
\KeywordTok{library}\NormalTok{(ggplot2)}
\KeywordTok{qplot}\NormalTok{(}\DataTypeTok{data=} \NormalTok{mtcars, }\DataTypeTok{x=}\NormalTok{wt)}
\end{Highlighting}
\end{Shaded}

\begin{verbatim}
## stat_bin: binwidth defaulted to range/30. Use 'binwidth = x' to adjust this.
\end{verbatim}

\includegraphics{./R-File_files/figure-latex/unnamed-chunk-15.pdf}

\subsection{Datentypen}\label{datentypen}

\begin{enumerate}
\def\labelenumi{\arabic{enumi}.}
\itemsep1pt\parskip0pt\parsep0pt
\item
  Vektoren
\end{enumerate}

Ein Beispiel für Vektoren

\begin{Shaded}
\begin{Highlighting}[]
\NormalTok{a <-}\StringTok{ }\KeywordTok{c}\NormalTok{(}\DecValTok{1}\NormalTok{,}\DecValTok{2}\NormalTok{,}\FloatTok{5.3}\NormalTok{,}\DecValTok{6}\NormalTok{,-}\DecValTok{2}\NormalTok{,}\DecValTok{4}\NormalTok{) }\CommentTok{# numeric vector}
\NormalTok{b <-}\StringTok{ }\KeywordTok{c}\NormalTok{(}\StringTok{"one"}\NormalTok{,}\StringTok{"two"}\NormalTok{,}\StringTok{"three"}\NormalTok{) }\CommentTok{# character vector}
\NormalTok{c <-}\StringTok{ }\KeywordTok{c}\NormalTok{(}\OtherTok{TRUE}\NormalTok{,}\OtherTok{TRUE}\NormalTok{,}\OtherTok{TRUE}\NormalTok{,}\OtherTok{FALSE}\NormalTok{,}\OtherTok{TRUE}\NormalTok{,}\OtherTok{FALSE}\NormalTok{) }\CommentTok{#logical vector}
\end{Highlighting}
\end{Shaded}

\begin{enumerate}
\def\labelenumi{\arabic{enumi}.}
\setcounter{enumi}{1}
\itemsep1pt\parskip0pt\parsep0pt
\item
  Matrizen
\end{enumerate}

Ein Beispiel für Matrizen

\begin{Shaded}
\begin{Highlighting}[]
\CommentTok{# generates 5 x 4 numeric matrix }
\NormalTok{y<-}\KeywordTok{matrix}\NormalTok{(}\DecValTok{1}\NormalTok{:}\DecValTok{20}\NormalTok{, }\DataTypeTok{nrow=}\DecValTok{5}\NormalTok{,}\DataTypeTok{ncol=}\DecValTok{4}\NormalTok{)}

\CommentTok{# another example}
\NormalTok{cells <-}\StringTok{ }\KeywordTok{c}\NormalTok{(}\DecValTok{1}\NormalTok{,}\DecValTok{26}\NormalTok{,}\DecValTok{24}\NormalTok{,}\DecValTok{68}\NormalTok{)}
\NormalTok{rnames <-}\StringTok{ }\KeywordTok{c}\NormalTok{(}\StringTok{"R1"}\NormalTok{, }\StringTok{"R2"}\NormalTok{)}
\NormalTok{cnames <-}\StringTok{ }\KeywordTok{c}\NormalTok{(}\StringTok{"C1"}\NormalTok{, }\StringTok{"C2"}\NormalTok{) }
\NormalTok{mymatrix <-}\StringTok{ }\KeywordTok{matrix}\NormalTok{(cells, }\DataTypeTok{nrow=}\DecValTok{2}\NormalTok{, }\DataTypeTok{ncol=}\DecValTok{2}\NormalTok{, }\DataTypeTok{byrow=}\OtherTok{TRUE}\NormalTok{,}
  \DataTypeTok{dimnames=}\KeywordTok{list}\NormalTok{(rnames, cnames))}
\NormalTok{mymatrix[]}
\end{Highlighting}
\end{Shaded}

\begin{verbatim}
##    C1 C2
## R1  1 26
## R2 24 68
\end{verbatim}

\begin{enumerate}
\def\labelenumi{(\arabic{enumi})}
\setcounter{enumi}{2}
\itemsep1pt\parskip0pt\parsep0pt
\item
  Arrays
\end{enumerate}

Sind wie Matrizen aufgebaut, nur sind mehrere Dimensionen möglich Arrays
are similar to matrices but can have more than two dimensions. See
help(array) for details.

\begin{enumerate}
\def\labelenumi{(\arabic{enumi})}
\setcounter{enumi}{3}
\itemsep1pt\parskip0pt\parsep0pt
\item
  Data Frames
\end{enumerate}

A data frame is more general than a matrix, in that different columns
can have different modes (numeric, character, factor, etc.). This is
similar to SAS and SPSS datasets.

\begin{Shaded}
\begin{Highlighting}[]
\NormalTok{d <-}\StringTok{ }\KeywordTok{c}\NormalTok{(}\DecValTok{1}\NormalTok{,}\DecValTok{2}\NormalTok{,}\DecValTok{3}\NormalTok{,}\DecValTok{4}\NormalTok{)}
\NormalTok{e <-}\StringTok{ }\KeywordTok{c}\NormalTok{(}\StringTok{"red"}\NormalTok{, }\StringTok{"white"}\NormalTok{, }\StringTok{"red"}\NormalTok{, }\OtherTok{NA}\NormalTok{)}
\NormalTok{f <-}\StringTok{ }\KeywordTok{c}\NormalTok{(}\OtherTok{TRUE}\NormalTok{,}\OtherTok{TRUE}\NormalTok{,}\OtherTok{TRUE}\NormalTok{,}\OtherTok{FALSE}\NormalTok{)}
\NormalTok{mydata <-}\StringTok{ }\KeywordTok{data.frame}\NormalTok{(d,e,f)}
\KeywordTok{names}\NormalTok{(mydata) <-}\StringTok{ }\KeywordTok{c}\NormalTok{(}\StringTok{"ID"}\NormalTok{,}\StringTok{"Color"}\NormalTok{,}\StringTok{"Passed"}\NormalTok{) }\CommentTok{# variable names}
\NormalTok{mydata}
\end{Highlighting}
\end{Shaded}

\begin{verbatim}
##   ID Color Passed
## 1  1   red   TRUE
## 2  2 white   TRUE
## 3  3   red   TRUE
## 4  4  <NA>  FALSE
\end{verbatim}

\begin{enumerate}
\def\labelenumi{\arabic{enumi}.}
\setcounter{enumi}{4}
\itemsep1pt\parskip0pt\parsep0pt
\item
  List
\end{enumerate}

An ordered collection of objects (components). A list allows you to
gather a variety of (possibly unrelated) objects under one name.

\begin{Shaded}
\begin{Highlighting}[]
\CommentTok{# example of a list with 4 components - }
\CommentTok{# a string, a numeric vector, a matrix, and a scaler }
\NormalTok{w <-}\StringTok{ }\KeywordTok{list}\NormalTok{(}\DataTypeTok{name=}\StringTok{"Fred"}\NormalTok{, }\DataTypeTok{mynumbers=}\NormalTok{a, }\DataTypeTok{mymatrix=}\NormalTok{y, }\DataTypeTok{age=}\FloatTok{5.3}\NormalTok{)}
\end{Highlighting}
\end{Shaded}

\begin{verbatim}
# example of a list containing two lists 
v <- c(list1,list2)
\end{verbatim}

\begin{enumerate}
\def\labelenumi{\arabic{enumi}.}
\setcounter{enumi}{5}
\itemsep1pt\parskip0pt\parsep0pt
\item
  Factors
\end{enumerate}

Tell R that a variable is nominal by making it a factor. The factor
stores the nominal values as a vector of integers in the range {[}
1\ldots{} k {]} (where k is the number of unique values in the nominal
variable), and an internal vector of character strings (the original
values) mapped to these integers.

\begin{itemize}
\itemsep1pt\parskip0pt\parsep0pt
\item
  Nominale Variablen
\end{itemize}

\begin{Shaded}
\begin{Highlighting}[]
\CommentTok{# variable gender with 20 "male" entries and }
\CommentTok{# 30 "female" entries }
\NormalTok{gender <-}\StringTok{ }\KeywordTok{c}\NormalTok{(}\KeywordTok{rep}\NormalTok{(}\StringTok{"male"}\NormalTok{,}\DecValTok{20}\NormalTok{), }\KeywordTok{rep}\NormalTok{(}\StringTok{"female"}\NormalTok{, }\DecValTok{30}\NormalTok{)) }
\NormalTok{gender <-}\StringTok{ }\KeywordTok{factor}\NormalTok{(gender) }
\CommentTok{# stores gender as 20 1s and 30 2s and associates}
\CommentTok{# 1=female, 2=male internally (alphabetically)}
\CommentTok{# R now treats gender as a nominal variable }
\KeywordTok{summary}\NormalTok{(gender)}
\end{Highlighting}
\end{Shaded}

\begin{verbatim}
## female   male 
##     30     20
\end{verbatim}

\begin{itemize}
\itemsep1pt\parskip0pt\parsep0pt
\item
  Ordinale Variablen
\end{itemize}

An ordered factor is used to represent an ordinal variable.

\begin{Shaded}
\begin{Highlighting}[]
\CommentTok{# variable rating coded as "large", "medium", "small'}
\NormalTok{rating <-}\StringTok{ }\KeywordTok{c}\NormalTok{(}\KeywordTok{rep}\NormalTok{(}\StringTok{"large"}\NormalTok{), }\KeywordTok{rep}\NormalTok{(}\StringTok{"medium"}\NormalTok{), }\KeywordTok{rep}\NormalTok{(}\StringTok{"small"}\NormalTok{))}
\NormalTok{rating <-}\StringTok{ }\KeywordTok{ordered}\NormalTok{(rating)}

\KeywordTok{summary}\NormalTok{(rating)}
\end{Highlighting}
\end{Shaded}

\begin{verbatim}
##  large medium  small 
##      1      1      1
\end{verbatim}

\begin{Shaded}
\begin{Highlighting}[]
\CommentTok{# recodes rating to 1,2,3 and associates}
\CommentTok{# 1=large, 2=medium, 3=small internally}
\CommentTok{# R now treats rating as ordinal}
\end{Highlighting}
\end{Shaded}

R will treat factors as nominal variables and ordered factors as ordinal
variables in statistical proceedures and graphical analyses. You can use
options in the factor( ) and ordered( ) functions to control the mapping
of integers to strings (overiding the alphabetical ordering). You can
also use factors to create value labels. For more on factors see the
UCLA page.

\begin{Shaded}
\begin{Highlighting}[]
\KeywordTok{length}\NormalTok{(mtcars) }\CommentTok{# number of the variables or components}
\end{Highlighting}
\end{Shaded}

\begin{verbatim}
## [1] 12
\end{verbatim}

\begin{Shaded}
\begin{Highlighting}[]
\KeywordTok{str}\NormalTok{(mtcars)    }\CommentTok{# structure of an object }
\end{Highlighting}
\end{Shaded}

\begin{verbatim}
## 'data.frame':    32 obs. of  12 variables:
##  $ mpg         : num  21 21 22.8 21.4 18.7 18.1 14.3 24.4 22.8 19.2 ...
##  $ cyl         : num  6 6 4 6 8 6 8 4 4 6 ...
##  $ disp        : num  160 160 108 258 360 ...
##  $ hp          : num  110 110 93 110 175 105 245 62 95 123 ...
##  $ drat        : num  3.9 3.9 3.85 3.08 3.15 2.76 3.21 3.69 3.92 3.92 ...
##  $ wt          : num  2.62 2.88 2.32 3.21 3.44 ...
##  $ qsec        : num  16.5 17 18.6 19.4 17 ...
##  $ vs          : num  0 0 1 1 0 1 0 1 1 1 ...
##  $ am          : num  1 1 1 0 0 0 0 0 0 0 ...
##  $ gear        : num  4 4 4 3 3 3 3 4 4 4 ...
##  $ carb        : num  4 4 1 1 2 1 4 2 2 4 ...
##  $ weight_class: chr  "light" "light" "light" "average" ...
\end{verbatim}

\begin{Shaded}
\begin{Highlighting}[]
\KeywordTok{class}\NormalTok{(mtcars$wt)  }\CommentTok{# class or type of an object}
\end{Highlighting}
\end{Shaded}

\begin{verbatim}
## [1] "numeric"
\end{verbatim}

\begin{Shaded}
\begin{Highlighting}[]
\KeywordTok{class}\NormalTok{(mtcars) }\CommentTok{# many options}
\end{Highlighting}
\end{Shaded}

\begin{verbatim}
## [1] "data.frame"
\end{verbatim}

\begin{Shaded}
\begin{Highlighting}[]
\KeywordTok{names}\NormalTok{(mtcars) }\CommentTok{# names of the Variables}
\end{Highlighting}
\end{Shaded}

\begin{verbatim}
##  [1] "mpg"          "cyl"          "disp"         "hp"          
##  [5] "drat"         "wt"           "qsec"         "vs"          
##  [9] "am"           "gear"         "carb"         "weight_class"
\end{verbatim}

\begin{Shaded}
\begin{Highlighting}[]
\NormalTok{mtcars     }\CommentTok{# prints the object mtcars}
\end{Highlighting}
\end{Shaded}

\begin{verbatim}
##                      mpg cyl  disp  hp drat    wt  qsec vs am gear carb
## Mazda RX4           21.0   6 160.0 110 3.90 2.620 16.46  0  1    4    4
## Mazda RX4 Wag       21.0   6 160.0 110 3.90 2.875 17.02  0  1    4    4
## Datsun 710          22.8   4 108.0  93 3.85 2.320 18.61  1  1    4    1
## Hornet 4 Drive      21.4   6 258.0 110 3.08 3.215 19.44  1  0    3    1
## Hornet Sportabout   18.7   8 360.0 175 3.15 3.440 17.02  0  0    3    2
## Valiant             18.1   6 225.0 105 2.76 3.460 20.22  1  0    3    1
## Duster 360          14.3   8 360.0 245 3.21 3.570 15.84  0  0    3    4
## Merc 240D           24.4   4 146.7  62 3.69 3.190 20.00  1  0    4    2
## Merc 230            22.8   4 140.8  95 3.92 3.150 22.90  1  0    4    2
## Merc 280            19.2   6 167.6 123 3.92 3.440 18.30  1  0    4    4
## Merc 280C           17.8   6 167.6 123 3.92 3.440 18.90  1  0    4    4
## Merc 450SE          16.4   8 275.8 180 3.07 4.070 17.40  0  0    3    3
## Merc 450SL          17.3   8 275.8 180 3.07 3.730 17.60  0  0    3    3
## Merc 450SLC         15.2   8 275.8 180 3.07 3.780 18.00  0  0    3    3
## Cadillac Fleetwood  10.4   8 472.0 205 2.93 5.250 17.98  0  0    3    4
## Lincoln Continental 10.4   8 460.0 215 3.00 5.424 17.82  0  0    3    4
## Chrysler Imperial   14.7   8 440.0 230 3.23 5.345 17.42  0  0    3    4
## Fiat 128            32.4   4  78.7  66 4.08 2.200 19.47  1  1    4    1
## Honda Civic         30.4   4  75.7  52 4.93 1.615 18.52  1  1    4    2
## Toyota Corolla      33.9   4  71.1  65 4.22 1.835 19.90  1  1    4    1
## Toyota Corona       21.5   4 120.1  97 3.70 2.465 20.01  1  0    3    1
## Dodge Challenger    15.5   8 318.0 150 2.76 3.520 16.87  0  0    3    2
## AMC Javelin         15.2   8 304.0 150 3.15 3.435 17.30  0  0    3    2
## Camaro Z28          13.3   8 350.0 245 3.73 3.840 15.41  0  0    3    4
## Pontiac Firebird    19.2   8 400.0 175 3.08 3.845 17.05  0  0    3    2
## Fiat X1-9           27.3   4  79.0  66 4.08 1.935 18.90  1  1    4    1
## Porsche 914-2       26.0   4 120.3  91 4.43 2.140 16.70  0  1    5    2
## Lotus Europa        30.4   4  95.1 113 3.77 1.513 16.90  1  1    5    2
## Ford Pantera L      15.8   8 351.0 264 4.22 3.170 14.50  0  1    5    4
## Ferrari Dino        19.7   6 145.0 175 3.62 2.770 15.50  0  1    5    6
## Maserati Bora       15.0   8 301.0 335 3.54 3.570 14.60  0  1    5    8
## Volvo 142E          21.4   4 121.0 109 4.11 2.780 18.60  1  1    4    2
##                     weight_class
## Mazda RX4                  light
## Mazda RX4 Wag              light
## Datsun 710                 light
## Hornet 4 Drive           average
## Hornet Sportabout        average
## Valiant                  average
## Duster 360                 heavy
## Merc 240D                average
## Merc 230                 average
## Merc 280                 average
## Merc 280C                average
## Merc 450SE                 heavy
## Merc 450SL                 heavy
## Merc 450SLC                heavy
## Cadillac Fleetwood         heavy
## Lincoln Continental        heavy
## Chrysler Imperial          heavy
## Fiat 128                   light
## Honda Civic                light
## Toyota Corolla             light
## Toyota Corona              light
## Dodge Challenger           heavy
## AMC Javelin              average
## Camaro Z28                 heavy
## Pontiac Firebird           heavy
## Fiat X1-9                  light
## Porsche 914-2              light
## Lotus Europa               light
## Ford Pantera L           average
## Ferrari Dino               light
## Maserati Bora              heavy
## Volvo 142E                 light
\end{verbatim}

\begin{Shaded}
\begin{Highlighting}[]
\KeywordTok{ls}\NormalTok{()   }\CommentTok{# list current objects}
\end{Highlighting}
\end{Shaded}

\begin{verbatim}
##  [1] "a"        "b"        "c"        "cells"    "cnames"   "d"       
##  [7] "e"        "f"        "gender"   "mtcars"   "mydata"   "mymatrix"
## [13] "rating"   "rnames"   "w"        "y"
\end{verbatim}

\begin{verbatim}
c(object,object,...)       # combine objects into a vector
cbind(object, object, ...) # combine objects as columns
rbind(object, object, ...) # combine objects as rows 

newobject <- edit(object) # edit copy and save as newobject 
fix(object)               # edit in place
\end{verbatim}

\begin{tabular}{|l|l|}\hline
Age & Frequency \\ \hline
18--25  & 15 \\
26--35  & 33 \\
36--45  & 22 \\ \hline
\end{tabular}

\section{Tabellen}\label{tabellen}

Zum Einbinden von Tabellen eignet sich R-Markdown ebenfalls, ess muss
nur die Tabelle in der Datei haben ein Beispiel:

\begin{verbatim}
man kann die Layouts verändern, wenn man dies in die Meta daten der Datei schreibt

rmarkdown::tufte_handout:
  highlight: zenburn
  
wenn man die Folgenden Daten im Chunkout schreibt wird die Tabele im R ausgeführt.

library(xtable)
options(xtable.comment = FALSE)
options(xtable.booktabs = TRUE)
xtable(head(mtcars[, 1:6]), caption = "First rows of mtcars")
\end{verbatim}

You can also embed plots, for example:

\includegraphics{./R-File_files/figure-latex/unnamed-chunk-23.pdf}

Note that the \texttt{echo = FALSE} parameter was added to the code
chunk to prevent printing of the R code that generated the plot.
\#\#Funktionen für das Setup des Pdf´s unter R-Studion

\paragraph{So macht man links}\label{so-macht-man-links}

\begin{verbatim}
[Udacity website](https://www.udacity.com/course/viewer#!/c-ud651/l-729069797/e-804129319/m-811719066)
\end{verbatim}

\emph{This is an R Markdown document. Markdown is a simple formatting
syntax for authoring HTML, PDF, and MS Word documents. For more details
on using R Markdown see \url{http://rmarkdown.rstudio.com}.}

\end{document}
