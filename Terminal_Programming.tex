\documentclass[]{article}
\usepackage[T1]{fontenc}
\usepackage{lmodern}
\usepackage{amssymb,amsmath}
\usepackage{ifxetex,ifluatex}
\usepackage{fixltx2e} % provides \textsubscript
% use upquote if available, for straight quotes in verbatim environments
\IfFileExists{upquote.sty}{\usepackage{upquote}}{}
\ifnum 0\ifxetex 1\fi\ifluatex 1\fi=0 % if pdftex
  \usepackage[utf8]{inputenc}
\else % if luatex or xelatex
  \ifxetex
    \usepackage{mathspec}
    \usepackage{xltxtra,xunicode}
  \else
    \usepackage{fontspec}
  \fi
  \defaultfontfeatures{Mapping=tex-text,Scale=MatchLowercase}
  \newcommand{\euro}{€}
\fi
% use microtype if available
\IfFileExists{microtype.sty}{\usepackage{microtype}}{}
\usepackage[margin=1in]{geometry}
\ifxetex
  \usepackage[setpagesize=false, % page size defined by xetex
              unicode=false, % unicode breaks when used with xetex
              xetex]{hyperref}
\else
  \usepackage[unicode=true]{hyperref}
\fi
\hypersetup{breaklinks=true,
            bookmarks=true,
            pdfauthor={},
            pdftitle={Terminal Funktionen},
            colorlinks=true,
            citecolor=blue,
            urlcolor=blue,
            linkcolor=magenta,
            pdfborder={0 0 0}}
\urlstyle{same}  % don't use monospace font for urls
\setlength{\parindent}{0pt}
\setlength{\parskip}{6pt plus 2pt minus 1pt}
\setlength{\emergencystretch}{3em}  % prevent overfull lines
\setcounter{secnumdepth}{5}

%%% Change title format to be more compact
\usepackage{titling}
\setlength{\droptitle}{-2em}
  \title{Terminal Funktionen}
  \pretitle{\vspace{\droptitle}\centering\huge}
  \posttitle{\par}
  \author{}
  \preauthor{}\postauthor{}
  \date{}
  \predate{}\postdate{}




\begin{document}

\maketitle


{
\hypersetup{linkcolor=black}
\setcounter{tocdepth}{2}
\tableofcontents
}
\begin{center}\rule{0.5\linewidth}{\linethickness}\end{center}

\section{Teriminal Notes}\label{teriminal-notes}

Hier kommen die Notizen zu meinen Terminal Eingaben, die ich bislang
kann.

\begin{itemize}
\itemsep1pt\parskip0pt\parsep0pt
\item
  \texttt{pwd} -- to see what the current working directory is.
\item
  \texttt{ls} -- um die Unterverzeichnise bzw. Dateien anzuzeigen.
\item
  \texttt{cd foo} -- to change to the foo subdirectory of your working
  directory.
\item
  \texttt{cd ..} -- to move up to the parent of the working directory.
\item
  \texttt{mkdir foo} -- to create a subdirectory called foo in the
  working directory.
\item
  \texttt{pandoc -\/-version} -- um zu erfahren welche Pandoc-Version
  installiert ist
\end{itemize}

\begin{center}\rule{0.5\linewidth}{\linethickness}\end{center}

\textbf{um eine Datei durch pandoc in eine html-Datei zuschreiben}

\begin{verbatim}
pandoc test1.md -f markdown -t html -s -o test1.html
\end{verbatim}

\textbf{um eine Datei in eine pdf datei umzuwandeln}

\begin{verbatim}
pandoc text.md -s -o test1.pdf
\end{verbatim}

\textbf{falls man eine Option/Funktion vergessen hat -- hilft
folgendes:}

\begin{verbatim}
pandoc --help
-- oder -- 
man pandoc # ruft das Manual auf
\end{verbatim}

\section{Shortcuts}\label{shortcuts}

\begin{itemize}
\itemsep1pt\parskip0pt\parsep0pt
\item
  \texttt{up-arrow} -- to go back through your command history.
\item
  mit dem \texttt{Tab} werden Komandos vervollständigt.
\item
  mit \texttt{cmd d} lässt sich das Terminal splitten
\item
  mit \texttt{Ctrl-D}kommt man aus einem Programm raus
\item
  \texttt{ctrl-c} -- zum beenden von laufenden Programmen
\end{itemize}

\end{document}
