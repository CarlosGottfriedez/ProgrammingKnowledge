\documentclass[]{article}
\usepackage[T1]{fontenc}
\usepackage{lmodern}
\usepackage{amssymb,amsmath}
\usepackage{ifxetex,ifluatex}
\usepackage{fixltx2e} % provides \textsubscript
% use upquote if available, for straight quotes in verbatim environments
\IfFileExists{upquote.sty}{\usepackage{upquote}}{}
\ifnum 0\ifxetex 1\fi\ifluatex 1\fi=0 % if pdftex
  \usepackage[utf8]{inputenc}
\else % if luatex or xelatex
  \ifxetex
    \usepackage{mathspec}
    \usepackage{xltxtra,xunicode}
  \else
    \usepackage{fontspec}
  \fi
  \defaultfontfeatures{Mapping=tex-text,Scale=MatchLowercase}
  \newcommand{\euro}{€}
\fi
% use microtype if available
\IfFileExists{microtype.sty}{\usepackage{microtype}}{}
\usepackage[margin=1in]{geometry}
\ifxetex
  \usepackage[setpagesize=false, % page size defined by xetex
              unicode=false, % unicode breaks when used with xetex
              xetex]{hyperref}
\else
  \usepackage[unicode=true]{hyperref}
\fi
\hypersetup{breaklinks=true,
            bookmarks=true,
            pdfauthor={},
            pdftitle={Terminal Funktionen},
            colorlinks=true,
            citecolor=blue,
            urlcolor=blue,
            linkcolor=magenta,
            pdfborder={0 0 0}}
\urlstyle{same}  % don't use monospace font for urls
\setlength{\parindent}{0pt}
\setlength{\parskip}{6pt plus 2pt minus 1pt}
\setlength{\emergencystretch}{3em}  % prevent overfull lines
\setcounter{secnumdepth}{5}

%%% Change title format to be more compact
\usepackage{titling}
\setlength{\droptitle}{-2em}
  \title{Terminal Funktionen}
  \pretitle{\vspace{\droptitle}\centering\huge}
  \posttitle{\par}
  \author{}
  \preauthor{}\postauthor{}
  \date{}
  \predate{}\postdate{}




\begin{document}

\maketitle


{
\hypersetup{linkcolor=black}
\setcounter{tocdepth}{2}
\tableofcontents
}
\begin{center}\rule{0.5\linewidth}{\linethickness}\end{center}

\section{Teriminal Notes}\label{teriminal-notes}

Hier kommen die Notizen zu meinen Terminal Eingaben, die ich bislang
kann.

\begin{itemize}
\itemsep1pt\parskip0pt\parsep0pt
\item
  \texttt{pwd} -- to see what the current working directory is.
\item
  \texttt{ls} -- um die Unterverzeichnise bzw. Dateien anzuzeigen.
\item
  \texttt{cd foo} -- to change to the foo subdirectory of your working
  directory.
\item
  \texttt{cd ..} -- to move up to the parent of the working directory.
\item
  \texttt{mkdir foo} -- to create a subdirectory called foo in the
  working directory.
\item
  \texttt{pandoc -\/-version} -- um zu erfahren welche Pandoc-Version
  installiert ist
\end{itemize}

\begin{center}\rule{0.5\linewidth}{\linethickness}\end{center}

\textbf{um eine Datei durch pandoc in eine html-Datei zuschreiben}

\begin{verbatim}
pandoc test1.md -f markdown -t html -s -o test1.html
\end{verbatim}

\textbf{um eine Datei in eine pdf datei umzuwandeln}

\begin{verbatim}
pandoc text.md -s -o test1.pdf
\end{verbatim}

\textbf{falls man eine Option/Funktion vergessen hat -- hilft
folgendes:}

\begin{verbatim}
pandoc --help
-- oder -- 
man pandoc # ruft das Manual auf
\end{verbatim}

\section{Shortcuts}\label{shortcuts}

\begin{itemize}
\itemsep1pt\parskip0pt\parsep0pt
\item
  \texttt{up-arrow} -- to go back through your command history.
\item
  mit dem \texttt{Tab} werden Komandos vervollständigt.
\item
  mit \texttt{cmd d} lässt sich das Terminal splitten
\item
  mit \texttt{Ctrl-D}kommt man aus einem Programm raus
\item
  \texttt{ctrl-c} -- zum beenden von laufenden Programmen
\end{itemize}

\subsection{Git-Funktionen}\label{git-funktionen}

\begin{verbatim}
q  :q  Q  :Q  ZZ     Exit.
\end{verbatim}

\begin{verbatim}
cd my-project
git init
\end{verbatim}

\begin{verbatim}
git add project.tex
\end{verbatim}

\begin{verbatim}
git add figures/     # Adds the figures directory
git add *.tex        # Adds all .tex files in the current directory
git add .            # Adds all of the files (. is the current directory)
\end{verbatim}

\begin{verbatim}
git commit -m "My first commit"
\end{verbatim}

You need to run git add whenever you change a file, not just at the
beginning. This tells git that it should include these changes in the
next snapshot. Alternatively you can pass the -a option to git commit to
tell it to commit all changes to files that it is monitoring:

\begin{verbatim}
git commit -a -m "My second commit"
\end{verbatim}

\begin{verbatim}
git log
\end{verbatim}

\begin{verbatim}
git show
\end{verbatim}

\begin{verbatim}
git checkout f69606d7e24ad45b31bb6eb4b38192bd07f274fc # Frühere verison mit dieser Nummer wird aufgerufen.
\end{verbatim}

Another common situation is you decide that you want to undo only one
set of changes from a while ago. Perhaps you've edited the document in
three steps: 1) Adding an abstract, 2) Updating your acknowledgements,
3) Adding in a figure. At each step you've created a new version in git,
but now you decide that you didn't really want to update you
acknowledgements. Unfortunately this is sandwiched between other changes
so a simple rollback like before won't do. Fear not, because git is
clever enough to do what you want, with the revert command:

\begin{verbatim}
git revert f69606d7e24ad45b31bb6eb4b38192bd07f274fc
\end{verbatim}

\begin{verbatim}
git push origin master
\end{verbatim}

The other possiblity is you want to download changes made by someone
else. This is a process which git calls `pulling'. To do this, simply
run

\begin{verbatim}
git pull origin master
\end{verbatim}

\begin{verbatim}
git status
\end{verbatim}

\subsection{Coursera course}\label{coursera-course}

Struktur eines Terminal Kommandos *
\texttt{command flags argument}--\texttt{cp -r Documents More\_docs} ein
flag wird von \texttt{-} eingeleitet.

\begin{itemize}
\itemsep1pt\parskip0pt\parsep0pt
\item
  \texttt{/} -- Root directorie
\item
  \texttt{\textasciitilde{}}-- home directorie
\item
  \texttt{clear} --macht alles wieder leer
\item
  \texttt{ls -a}- zeigt auch die versteckten Dateien
\item
  \texttt{ls -al}- zeigt die details von versteckten und nicht
  versteckten datein
\item
  \texttt{cd} -- direktory ändern, oder bring einem zum home direktor
\item
  \texttt{mkdir} -- + name erstellt einen neuen ordner
\item
  \texttt{touch test\_file} -- erzeugt ein neue Datei test\_file
\item
  \texttt{cp test\_file Documents} -- macht eine kopie vom test\_file in
  den Ordner Documents
\item
  \texttt{cp -r Documents More\_docs} -- kopiert ein Ordner Documents in
  einen andernOrdner More\_docs
\item
  \texttt{rm test\_file} -- löschen datei
\item
  \texttt{rm More\_docs} -- löscht ordner und alle Datein darin
\item
  \texttt{mv new\_file Documents} -- bewegt new\_file zu Documents
\item
  \texttt{mv new\_file renamed\_file} -- ändert den namen von new\_file
  zu renamed\_file
\item
  \texttt{echo Hallo welt} -- zeigt hallo welt
\item
  \texttt{date} -- zeigt datum
\end{itemize}

\subsection{Git}\label{git}

\href{http://gitready.com/}{Ein link}

\begin{itemize}
\itemsep1pt\parskip0pt\parsep0pt
\item
  \texttt{git add} -- alle neuen Dateien werden getracked von version
  control
\end{itemize}

\end{document}
